% !TEX TS-program = pdflatex
% !TEX encoding = UTF-8 Unicode

\documentclass[10pt]{article} % use larger type; default would be 10pt
\usepackage[utf8]{inputenc} % set input encoding (not needed with XeLaTeX)

%%% PAGE DIMENSIONS
\usepackage[margin=1in]{geometry} % to change the page dimensions
\geometry{a4paper} % or letterpaper (US) or a5paper or....
\usepackage{graphicx} % support the \includegraphics command and options
\usepackage[parfill]{parskip} % Activate to begin paragraphs with an empty line rather than an indent

%%% PACKAGES
\usepackage{amsmath}
\usepackage{booktabs} % for much better looking tables
\usepackage{array} % for better arrays (eg matrices) in maths
\usepackage{paralist} % very flexible & customisable lists (eg. enumerate/itemize, etc.)
\usepackage{verbatim} % adds environment for commenting out blocks of text & for better verbatim
\usepackage{subfig} % make it possible to include more than one captioned figure/table in a single float
% These packages are all incorporated in the memoir class to one degree or another...

%%% HEADERS & FOOTERS
\usepackage{fancyhdr} % This should be set AFTER setting up the page geometry
\pagestyle{fancy} % options: empty , plain , fancy
\renewcommand{\headrulewidth}{0pt} % customise the layout...
\lhead{}\chead{}\rhead{}
\lfoot{}\cfoot{\thepage}\rfoot{}

%%% SECTION TITLE APPEARANCE
\usepackage{sectsty}
\allsectionsfont{\sffamily\mdseries\upshape} % (See the fntguide.pdf for font help)
% (This matches ConTeXt defaults)

%%% ToC (table of contents) APPEARANCE
\usepackage[nottoc,notlof,notlot]{tocbibind} % Put the bibliography in the ToC
\usepackage[titles,subfigure]{tocloft} % Alter the style of the Table of Contents
\renewcommand{\cftsecfont}{\rmfamily\mdseries\upshape}
\renewcommand{\cftsecpagefont}{\rmfamily\mdseries\upshape} % No bold!

%%% Maths Stuff
\usepackage{pgfplots}
\pgfplotsset{compat = newest}

%%% END Article customizations

%%% ADMIN MACROS
\newcommand{\topic}[1]{\part[{#1}]{{#1}}}
\newcommand{\subtopic}[1]{\section[#1]{{#1}}}
\newcommand{\parttitle}[1]{\subsection[#1]{{#1}}}
\newcommand{\partsubtitle}[1]{\paragraph{{#1}}}

\newenvironment{example}[2]
{\begin{center}
		Example #1: #2 \\[1ex]
		\begin{tabular}{|p{0.9\textwidth}|}
			\hline\\
		}
		{ 
			\\\\\hline
		\end{tabular} 
	\end{center}
}

\newcommand{\nb}[1]{\textit{(NB: {#1})}}

%%% SPEED MACROS
\newcommand{\plotgraph}[7]{ %xmin xmax ymin ymax xtick ytick func
	\newline
	\begin{tikzpicture}
		\begin{axis}[
			xmin = {#1}, xmax = {#2},
			ymin = {#3}, ymax = {#4},
			xtick distance = {#5},
			ytick distance = {#6},
			grid = both,
			minor tick num = 1,
			major grid style = {lightgray},
			minor grid style = {lightgray!25},
			width = \textwidth,
			height = \textwidth / 2]
			]
			\addplot[] {{#7}};
		\end{axis}
	\end{tikzpicture}
	\newline
}
\newcommand{\plotbasic}[1]{
	\plotgraph{-5}{5}{-5}{25}{1}{2.5}{{#1}}
}

\title{Pure Mathematics Year 1/AS}
\author{Jack Maguire}
\date{}

\begin{document}
\maketitle

\tableofcontents
\newpage

\topic{Graphs and Transformations}	
\subtopic{Transformations}

The graph of \(f(x)\), here as an example: \(y = x^2\).
\plotbasic{x^2}


\parttitle{Translation}
\textbf{The graph of \(f(x-a)\) is the graph of \(f(x)\) translated right by \(a\) units.} For example, if we want to move our graph 2 units right, then we can plot \(y = (x-2)^2\).
\plotbasic{(x-2)^2}

\textbf{The graph of \(f(x)+b\) is the graph of \(f(x)\) translated upwards by \(b\) units.} For example, if we want to move our graph 3 units down, then we can plot \(y = x^2 - 3\).
\plotbasic{x^2-3}


\parttitle{Scaling}
\nb{Never say shrink - always say stretch by a factor \(e\) where \(|e| < 1\)}

\textbf{The graph of \(cf(x)\) is the graph of \(f(x)\) stretched vertically by a factor of \(c\).} For example, if we want to stretch our graph by a factor of 2, then we can plot \(y = 2 * x^2\).
\plotbasic{2 * x^2}

\textbf{The graph of \(f(dx)\) is the graph of \(f(x)\) stretched horizontally by a factor of \(d^{-1}\).} For example, if we want to stretch our graph by a factor of 5, then we can plot \(y = (0.2 * x)^2\).
\plotbasic{(0.2*x)^2}

\newcounter{TransformationsCounter}
\setcounter{TransformationsCounter}{1}

\begin{example}{\arabic{TransformationsCounter}}{Combining Transformations}
	
	\[
		y = cf(\frac{1}{a} * (x-b)) + d
	\]
	
	This is obtained from \(f(x)\) by doing the following:
	\begin{enumerate}
		\item Shift from the right \(b\) units.
		\item Stretch horizontally by a factor of \(a\).
		\item Stretch vertically by a factor of \(c\).
		\item Shift upwards by \(d\) units.
	\end{enumerate}
	
\end{example}
\addtocounter{TransformationsCounter}{1}
\begin{example}{\arabic{TransformationsCounter}}{Combining Transformations}
	
	\[
	y = f(-2x)
	\]
	
	This is obtained from \(f(x)\) by doing the following:
	\begin{enumerate}
		\item Flip horizontally.
		\item Stretch horizontally by a factor of \(0.5\).
	\end{enumerate}
	
\end{example}
\addtocounter{TransformationsCounter}{1}


\end{document}
